\newcount\draft\draft=1 % set to 0 for "publication"

\documentclass{article}
\usepackage[a4paper]{geometry}
\usepackage[utf8]{inputenc}
\usepackage[english]{babel}

% \usepackage[in]{fullpage}

% \usepackage{natbib}
\bibliographystyle{abbrv}

\usepackage{enumitem}

\usepackage[table,svgnames]{xcolor}
\usepackage{multicol}
\usepackage{graphicx}
\usepackage{pdflscape}

\usepackage{hyperref}
\hypersetup{
    colorlinks=true,
    linkcolor=FireBrick,
    citecolor=FireBrick,
    urlcolor=FireBrick
}

\usepackage{amsthm}
\usepackage{amssymb}
\usepackage{amsmath}

\usepackage[linesnumbered]{algorithm2e}
\let\oldnl\nl
\newcommand{\nonl}{\renewcommand{\nl}{\let\nl\oldnl}}

\setlength\parindent{0pt}

% \newcommand{\bl}[2][c]{%
%   \begin{tabular}[#1]{@{ }c@{ }}#2\end{tabular}}

\newcommand{\this}{\cellcolor{BlanchedAlmond}}
\newcommand{\that}{\cellcolor{Gainsboro}}
\newcommand{\look}{\color{FireBrick}}
\newcommand{\cell}[4] {{
  \begin{tabular}{cc}
    {\color{black!40}#1} & {\color{black!40}#2} \\
    {\color{black!40}#3} & #4 \\
  \end{tabular}
}}

\newtheorem{theorem}{Theorem}[section]
\newtheorem*{remark}{Remark}
\newtheorem{lemma}{Lemma}
\newtheorem{corollary}[theorem]{Corollary}
\newtheorem{property}{Property}
\newtheorem{proposition}{Proposition}
\newtheorem{fact}{Fact}

\renewcommand{\thefootnote}{\fnsymbol{footnote}}

\usepackage{cleveref} % MUST BE LOADED LAST

\title{Advanced Algorithms for Data Science \\ \textsc{Homework 3}}
\author{Krikun Gosha}
\date{ }

\begin{document}

\maketitle
\section{Dynamic programming}
\paragraph{Question}
\textit{Propose an algorithm to compute the optimal shipping schedule, that is
  an optimal sequence of n choices between A and B generating the minimal
  cost.}\\ 

This exercise could be reduced to Interval Scheduling Problem. All we need is in
correct way describe optimal solution function:

$$
OPT(i) = \textit{min}
\begin{cases}
  OPT(i-1) + r \cdot s_i \\
  OPT(i-4) + c \cdot 4, \text{if } i \geq 4
\end{cases}
$$

and $OPT(0) = 0$. First case represent calculation cost by one week (based on
previous), second case ``replace'' optimal results of 4 previous weeks if it
have less cost. Second case we can calculate only in comparison with first one,
so for first three steps we have to conduct costs to compare with.\\
  
And for getting the final answer we should make backward induction, from the
end, to determine a sequence of optimal actions.  

\section{Protein sequence alignment}
\paragraph{Question}
\textit{Compute a maximal-score alignment of protein sequences under the scoring
  matrix BLOSUM62 and the indel penalty d = 8.}\\

Sequences:\\

\texttt{HEAGAWGHEE}\\
\texttt{PAWHEAE}\\

If I right get the question - we should calculate by hands (or by implementation
in some programming language). I think I will faster calculate by hands.

\begin{landscape}
  {
\setlength\tabcolsep{4pt}
  % \footnotesize
\begin{tabular}{c|c|c|c|c|c|c|c|c|c|c|c}

    &     &  H &   E &   A &   G &   A &   W &   G &   H &   E &   E \\
  \hline
    &   0 & \this -8 & \that -16 & -24 & -32 & -40 & -48 & -56 & -64 & -72 & -80 \\
  \hline
  P &  -8 & \cell{-2}{-16}{-16}{-2} & \this\cell{-9}{-24}{-10}{-9} & \that\cell{\look -17}{-32}{\look -17}{-17} & \that\cell{-26}{-40}{-25}{-25} & \cell{-33}{-48}{-33}{-33} & \cell{-44}{-56}{-41}{-41} & \cell{-50}{-64}{-49}{-49} & \cell{-58}{-72}{-57}{-57} & \cell{-65}{-80}{-65}{-65} & \cell{-73}{-88}{-73}{-73} \\
  \hline
  A & -16 & \cell{-10}{-10}{-24}{-10} & \cell{-3}{-17}{-18}{-3} & \this\cell{-5}{-25}{-11}{-5} & \this\cell{-17}{-33}{-13}{-13} & \this\cell{\look -21}{-41}{\look -21}{-21} & \cell{-36}{-49}{-29}{-29} & \cell{-41}{-57}{-37}{-37} & \cell{-51}{-65}{-45}{-45} & \cell{-58}{-73}{-53}{-53} & \cell{-66}{-81}{-61}{-61} \\
  \hline
  W & -24 & \cell{-18}{-18}{-32}{-18} & \cell{-13}{-11}{-26}{-11} & \cell{-6}{-13}{-19}{-6} & \cell{-7}{-21}{-18}{-7} & \cell{-16}{-29}{-15}{-15} & \this\cell{-10}{-37}{-25}{-10} & \that\cell{-31}{-45}{-18}{-18} & \cell{-39}{-53}{-26}{-26} & \cell{-48}{-61}{-34}{-34} & \cell{-56}{-69}{-42}{-42} \\
  \hline
  H & -32 & \cell{-16}{-26}{-40}{-16} & \cell{-18}{-19}{-24}{-18} & \cell{-13}{-14}{-26}{-13} & \cell{-8}{-15}{-21}{-8} & \cell{-9}{-23}{-16}{-9} & \cell{-17}{-18}{-17}{-17} & \this\cell{-12}{-26}{-25}{-12} & \that\cell{-10}{-34}{-20}{-10} & \cell{-26}{-42}{-18}{-18} & \cell{-34}{-50}{-26}{-26} \\
  \hline
  E & -40 & \cell{-32}{-24}{-48}{-24} & \cell{-11}{-26}{-32}{-11} & \cell{-19}{-21}{-19}{-19} & \cell{-15}{-16}{-27}{-15} & \cell{-9}{-17}{-23}{-9} & \cell{-14}{-25}{-17}{-14} & \cell{-19}{-20}{-22}{-19} & \this\cell{-12}{-18}{-27}{-12} & \that\cell{-5}{-26}{-20}{-5} & \cell{-13}{-32}{-13}{-13} \\
  \hline
  A & -48 & \cell{-42}{-32}{-56}{-32} & \cell{-25}{-19}{-40}{-19} & \cell{-7}{-27}{-27}{-7} & \cell{-19}{-23}{-15}{-15} & \cell{-11}{-17}{-23}{-11} & \cell{-12}{-22}{-19}{-12} & \cell{-14}{-27}{-20}{-14} & \cell{-21}{-20}{-22}{-20} & \this\cell{\look -13}{\look -13}{-28}{-13} & \cell{-6}{-21}{-21}{-6} \\
  \hline
  E & -56 & \cell{-48}{-40}{-64}{-40} & \cell{-27}{-27}{-48}{-27} & \cell{-20}{-15}{-35}{-15} & \cell{-9}{-23}{-23}{-9} & \cell{-16}{-19}{-17}{-16} & \cell{-14}{-20}{-24}{-14} & \cell{-14}{-22}{-22}{-14} & \cell{-14}{-28}{-22}{-14} & \cell{-15}{-21}{-22}{-15} & \this\cell{-8}{-14}{-23}{-8} \\
\end{tabular}
}\\[2em]

Each cell in table consists of 4 values (this will help us do backward induction):\\

\cell{
  $Score_{(i-1, j-1)} + s(S_i,T_i)$
}{
  $Score_{(i, j-1)} - d$
}{
  $Score_{(i-1, j)} - d$
}{
  maximal between all three
}\\[1em]

I done backward induction with coloring cells (different ways in different
colors) and stressed {\look forks} with red color.

So this sequences could be align with same score by 6 different ways:\\

\texttt{
  \begin{tabular}{cccccccccc}
  H & E & A & G & A & W & G & H & E & E \\
    &   & | &   &   & | &   &   &   & | \\
    & P & A &   &   & W & H & E & A & E \\
  \end{tabular}
}\\[1em]
Score: $-8 + -1 + 4 + -8 + -8 + 11 + -2 + 0 + -1 + 5 = -8$\\[1em]

\end{landscape}


\texttt{
  \begin{tabular}{ccccccccccc}
  H & E & A & G & A & W & G & H & E &   & E \\
    &   & | &   &   & | &   & | & | &   & | \\
    & P & A &   &   & W &   & H & E & A & E \\
  \end{tabular}
}\\[1em]
Score: $-8 + -1 + 4 + -8 + -8 + 11 + -8 + 8 + 5 + -8 + 5 = -8$\\[1em]

\texttt{
  \begin{tabular}{cccccccccc}
  H & E & A & G & A & W & G & H & E & E \\
    &   &   &   & | & | &   &   &   & | \\
    & P &   &   & A & W & H & E & A & E \\
  \end{tabular}
}\\[1em]
Score: $-8 + -1 + -8 + -8 + 4 + 11 + -2 + 0 + -1 + 5 = -8$\\[1em]

\texttt{
  \begin{tabular}{ccccccccccc}
  H & E & A & G & A & W & G & H & E &   & E \\
    &   &   &   & | & | &   & | & | &   & | \\
    & P &   &   & A & W &   & H & E & A & E \\
  \end{tabular}
}\\[1em]
Score: $-8 + -1 + -8 + -8 + 4 + 11 + -8 + 8 + 5 + -8 + 5 = -8$\\[1em]

\texttt{
  \begin{tabular}{cccccccccc}
  H & E & A & G & A & W & G & H & E & E \\
    &   &   &   & | & | &   &   &   & | \\
    &   & P &   & A & W & H & E & A & E \\
  \end{tabular}
}\\[1em]
Score: $-8 + -8 + -1 + -8 + 4 + 11 + -2 + 0 + -1 + 5 = -8$\\[1em]

\texttt{
  \begin{tabular}{ccccccccccc}
  H & E & A & G & A & W & G & H & E &   & E \\
    &   &   &   & | & | &   & | & | &   & | \\
    &   & P &   & A & W &   & H & E & A & E \\
  \end{tabular}
}\\[1em]
Score: $-8 + -8 + -1 + -8 + 4 + 11 + -8 + 8 + 5 + -8 + 5 = -8$\\[1em]

\section{Hidden Markov Models}
\paragraph{Question}
\textit{Finish the example of the Forward-Backward algorithm that we didn’t
  finish in class.}\\

Two hidden states: raining, not-raining\\
Probabilities to stay in the same state is 0.7, to change 0.3\\
Probabilities modelling the person's behaviour:\\

\begin{tabular}{|l|c|c|}
  \hline
    & umbrella & no umbrella \\
  \hline
  raining & 0.9 & 0.1 \\
  \hline
  not raining & 0.2 & 0.8 \\
  \hline
\end{tabular}

\end{document}