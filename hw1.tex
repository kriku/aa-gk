\newcount\draft\draft=1 % set to 0 for "publication"

\documentclass{article}
\usepackage[utf8]{inputenc}
\usepackage[english]{babel}

\usepackage[plain]{fullpage}

\usepackage{natbib}
\bibliographystyle{abbrv}

\usepackage{enumitem}

\usepackage[dvipsnames]{xcolor}
\usepackage{multicol}
\usepackage{graphicx}

\usepackage{hyperref}
\hypersetup{
    colorlinks=true,
    linkcolor=BrickRed,
    citecolor=BrickRed,
    urlcolor=BrickRed
}

\usepackage{amsthm}
\usepackage{amssymb}
\usepackage{amsmath}



\usepackage[linesnumbered]{algorithm2e}
\let\oldnl\nl
\newcommand{\nonl}{\renewcommand{\nl}{\let\nl\oldnl}}

\setlength\parindent{0pt}

\newtheorem{theorem}{Theorem}[section]
\newtheorem*{remark}{Remark}
\newtheorem{lemma}{Lemma}
\newtheorem{corollary}[theorem]{Corollary}
\newtheorem{property}{Property}
\newtheorem{proposition}{Proposition}
\newtheorem{fact}{Fact}

% TikZ ---------------------------------------------
\usepackage{tikz}
\usetikzlibrary{calc}
\usetikzlibrary{snakes}
\usetikzlibrary{trees}
\tikzstyle{vertex}=[circle,fill=black!0,minimum size=4pt,inner sep=0pt]
\tikzstyle{smallvertex}=[circle,fill=black!0,minimum size=3pt,inner sep=0pt]

\usepackage{cleveref} % MUST BE LOADED LAST

\title{Advanced Algorithms for Data Science \\ \textsc{Homework 1}}
\author{Krikun Gosha}
\date{2016-10-07}

\begin{document}

\maketitle

\section{Depth-first and Bread-first search}

\begin{theorem}
  In undirected connected graph $G=(V, E)$,
  if DFS and BFS produce same predecessor subgraph $G_\pi$,
  then $G=G_\pi$ ($V=V_\pi$, $E=E_\pi$).
\end{theorem}


\begin{proof}
  Firstly lets consider BFS \cite[page 595]{introtoalg} and DFS \cite[page 604]{introtoalg} algorithms.

\begin{multicols}{2}

\IncMargin{1em}
\begin{algorithm}[H]
  \SetAlgoNoEnd\SetAlgoNoLine
  \SetNlSkip{1.5em}
  \DontPrintSemicolon
  \SetKwProg{Fn}{Function}{}{}

  \nonl \textsc{BFS}$(G,s)$

  \ForEach{vertex $u \in G.V - \{s\}$}{
    $u.color = $ \textsc{white}\;
    $u.d = \infty$\;
    $u.\pi = nil$\;
  }

  $s.color = $ \textsc{gray}\;
  $s.d = 0$\;
  $s.\pi = nil$\;

  $Q = \emptyset$\;
  \textsc{Enqueue} $(Q,s)$\;
  \While {$Q \neq \emptyset$}{
    $u = $ \textsc{Dequeue} $(Q)$ \;
    \ForEach{$v \in G.Adj[u]$}{
      \If{$v.color == $ \textsc{white}}{
        $v.color = $ \textsc{gray}\;
        $v.d = v.d + 1$\;
        $v.\pi = u$\;
        \textsc{Enqueue} $(Q,v)$\;
      }
    }
    $v.color = $ \textsc{black}\;
  }
\end{algorithm}

\columnbreak

\begin{algorithm}[H]
  \SetAlgoNoEnd\SetAlgoNoLine
  \SetNlSkip{1.5em}
  \DontPrintSemicolon
  \SetKwProg{Fn}{Function}{}{}

  \nonl \textsc{DFS}$(G,s)$

  \ForEach{$u \in G.V$}{
    $u.color = $ \textsc{white}\;
    $u.\pi = nil$\;
  }
  $time = 0$\;
  \ForEach{$u \in G.V$}{
    \If{$u.color == $ \textsc{white}}{
      \textsc{Dfs-Visit(G,u)}
    }
  }
\end{algorithm}

\begin{algorithm}[H]
  \SetAlgoNoEnd\SetAlgoNoLine
  \SetNlSkip{1.5em}
  \DontPrintSemicolon
  \SetKwProg{Fn}{Function}{}{}

  \nonl \textsc{DFS-Visit}$(G,u)$

  $time = time + 1$\;
  $u.d = time$\;
  $u.color = $\textsc{Gray}\;

  \ForEach{$v \in G.Adj[u]$}{
    \If{$v.color == $ \textsc{white}}{
      $v.\pi = u$\;
      \textsc{Dfs-Visit(G,v)}\;
    }
  }
  $u.color = $\textsc{Black}\;
  $time = time + 1$\;
  $u.f = time$\;
\end{algorithm}

\end{multicols}


BFS produce subgraph, corresponding to the $\pi$ (predecessor) attributes.\\
More formally, for a graph $G=(V,E)$ with source $s$,
we define the \textbf{predecessor subgraph} of a bread-first search as\\

$G_\pi = (V_\pi, E_\pi)$, where\\
$V_\pi = \{v \in V: v.\pi \neq \textsc{nil}\} \cup {s}$\\
and\\
$E_\pi = \{(v.\pi,v) : v \in V_\pi - \{s\}\}$\\

As soon as graph $G$ is connected(given), thus predecessor subgraph of
bread-first search forms a \textbf{bread-first tree} for $G$.
\cite[by Lemma 22.6][page 601]{introtoalg}.\\

In this case $V_\pi = V$ (bread-first search reach all vertices of graph $G$).\\

Let $E_\textsc{Bfs} = E_\pi$ in bread-first search.\\

Unlike breadth-first search, whose predecessor subgraph forms a tree, the
predecessor subgraph produced by a depth-first search may be composed of several
trees, because the search may repeat from multiple sources. Therefore, we define
the \textbf{predecessor subgraph} of a depth-first search slightly differently from that
of a breadth-first search: we let\\

$G_\pi = (V, E_\pi)$, where\\
$E_\pi = \{(v.\pi,v) : v \in V_\pi \text{ and } v_\pi \neq \textsc{nil}\}$\\

The predecessor subgraph of depth-first search forms a \textbf{depth-first
forest} comprising several \textbf{depth-first trees}. But as soon as graph
$G$ is connected(given), thus predecessor subgraph forms exactly one \textbf{depth-first
tree}.\\

Let $E_\textsc{Dfs} = E_\pi$ in depth-first search.\\


By depth-first search we could classify the edges of the connected undirected graph $G=(V,E)$.
\cite[Classification of edges][page 609]{introtoalg}. In undirected graph
\textbf{forward} and \textbf{cross} edges never occurs. \cite[by Theorem
22.10][page 610]{introtoalg}.\\

Predecessor subgraph of bread-first search and predecessor subgraph of
depth-first search could be different only in one case, if:\\

$\exists v \in V : e_\textsc{Bfs} \neq e_\textsc{Dfs}$, where\\
$e_\textsc{Bfs} \in E_\textsc{Bfs}, e_\textsc{Bfs} = (v.\pi, v), v.\pi$ -
predecessor of $v$ in breadth-first search\\
$e_\textsc{Dfs} \in E_\textsc{Dfs}, e_\textsc{Dfs} = (v.\pi, v), v.\pi$ -
predecessor of $v$ in depth-first search\\

This situation could be only if in graph $G$ occurs \textbf{back} edges. But as
soon as \textsc{Bfs} and \textsc{Dfs} produce same trees, thus all edges in
$E_\textsc{Dfs}$ is \textbf{tree} edges and presented in both subgraphs
$E_\textsc{Bfs} = E_\textsc{Dfs} = E$.
Therefore graph $G = G_\pi$.
\end{proof}

\begin{remark}[Self-loop]
Except for one important remark. This is true for simple graphs. But in
general, undirected connected graph could have self-loops. This edges from
vertex to itself not present neither in the BFS nor in the DFS.
\end{remark}

\pagebreak
\section{Minimum spanning tree}

\begin{theorem}
  If in weighted undirected graph $G=(V, E, w)$,
  exists a path between $p$ and $q$ consisting entirely
  of edges whose cost is smaller than cost of edge $e=(p,q) \in E$, then $e$
  does not belong to a minimum spanning tree.
\end{theorem}

\begin{proof}

Firstly lets consider generic method, which grows the minimal spanning tree
one edge in a time. 
The generic method manages a set of edges A, maintaining the
following loop invariant:\\

Prior to each iteration, $A$ is a subset of some minimum spanning tree.\\

At each step, we determine an edge $(u,v)$ that we can add to $A$ without violating
this invariant, in the sense that $A \cup \{(u,v)\}$ is also a subset of a minimum spanning
tree. 
We call such an edge a \textbf{safe edge} for $A$, since we can add it safely to $A$ while
maintaining the invariant.\\

\IncMargin{1em}
\begin{algorithm}[H]
  \SetAlgoNoEnd\SetAlgoNoLine
  \SetNlSkip{1.5em}
  \DontPrintSemicolon

  \nonl \textsc{Generic-MST}$(G,w)$

  $A = \emptyset$\;
  \While {$A$ does not form a spanning tree}{
    find an edge $(u,v)$ that is safe for $A$\;
    $A = A \cup \{(u,v)\}$
  }
  \Return $A$\;
\end{algorithm}

\vspace{1em}

We use cut-and-paste method to determine safe edges and grow the spanning tree
\cite[Theorem 23.1]{introtoalg}.\\

Let $T$ be a minimum spanning tree that includes $A$.
Consider cut $(S, V - S)$, such that $p$ and $q$ on the opposite
sides of the cut. As soon as graph $G$ has simple path $\rho$ from $p$ to $q$,
$\rho \neq (p,q)$, entirely consisting of edges with less cost (given).
Thus cut cross some edge $e \in \rho$ with less cost than $(p,q)$, which will be
added in minimum spanning tree $T$, hereby on each iteration to $T$ added edges
from $\rho$. So path $\rho \in A$ and adding edge $(p,q)$ will produce
a cycle, what contradicts minimum spanning tree definition. Therefore edge
$(p,q) \notin T$.
\end{proof}

\begin{theorem}
  If in weighted undirected graph $G=(V, E, w)$,
  edge $e=(p,q) \in E$, does not belong to a minimum spanning tree,
  then there exists a path between $p$ and $q$ consisting entirely
  of edges whose cost is smaller or equal cost of $e$.
\end{theorem}

\begin{proof}
  Let $T$ be a minimum spanning tree that does not include edge $(p,q)$.
  Consider arbitrary cut $(S, V - S)$, such that $p$ and $q$ on the opposite
  sides of the cut. Since $(p,q)$ does not belong to a minimum spanning tree,
  thus cut cross some edge $e \in T$, such what, $w(e) \leq w(p,q)$. Therefore
  $\exists \rho = \{e_1, e_2 \dots e_n\} \cup \rho \neq (p,q) $,   
  such that $\forall e \in \rho : w(e) \leq w(p,q)$.
\end{proof}

\begin{remark}
  Theorem 2.2 claims that path $\rho$ could consists of edges with less or equal
  costs. There is a case in which $(p,q)$ does not belong to a minimal spanning
  tree, but simple path does not consists of edges which have less cost
  strictly, ie $\exists e \in \rho : w(e) = w(p,q)$ but in this ties edge
  $(p,q)$ does not chosen.
\end{remark}
  
\begin{corollary}
Using the above facts, we can propose an $O(m+n)$ algorithm that checks if a
given edge $e$ belongs to at least one minimum spanning tree.
\end{corollary}

We are interested in finding the simple path $\rho$ from $p$ to $q$, not equal
$\{(p,q)\}$ and consisting 
entirely from edges with strictly less weight (in case of equal, edge $(p,q)$ could
be presented in some MST). If this path exists, edge $(p,q)$ couldn't
belongs to any minimum spanning tree by Theorem 2.1.\\

For finding this path we could use bread-first search with some restrictions on
weight of the edge between adjacent vertices and not considering of edge $(p,q)$:\\

\IncMargin{1em}
\begin{algorithm}[H]
  \SetAlgoNoEnd\SetAlgoNoLine
  \SetNlSkip{1.5em}
  \DontPrintSemicolon

  \nonl \textsc{Check-MST}$(G,w,e)$\;
  $p = e.head$\;
  $q = e.tail$\;
  \ForEach{$u \in V - \{p\}$}{
    $u.color = \textsc{white}$\;
    $u.\pi = nil$\;
  }
  $Q = \emptyset$\;
  $p.color = \textsc{gray}$\;
  $p.\pi = \textsc{nil}$\;
  \textsc{Enqueue} $(Q,p)$\;
  \While {$Q \neq \emptyset$}{
    $u = $ \textsc{Dequeue} $(Q)$ \;
    \ForEach{$v \in G.Adj[u]$}{
      \If{$(v.color ==  \textsc{white}) \wedge (w(u,v) < w(p,q)) \wedge ((u,v) \neq (p,q))$}{
        $v.color = $ \textsc{gray}\;
        $v.\pi = u$\;
        \If{$v==q$}{
          \Return doesn't belong to any MST
          \tcp*{$\exists \rho=\{e_1, e_2 \dots e_n\} : \forall e \in \rho, w(e) < w(p,q)$}
        }
        \textsc{Enqueue} $(Q,v)$\;
      }
    }
    $v.color = $ \textsc{black}\;
  }
  \Return edge belongs to some MST\;
\end{algorithm}

\vspace{1em}
Consider algorithm complexity - while loop iterates maximum $|V|-1$ times (minus one for head
$p$ of edge). Nested loop ``for each'' iterates maximum $|E|-1$ times (one for $(p,q)$ edge).
So total complexity $O(|V|+|E|)$.


\pagebreak
\section{Network flow}

\begin{theorem}
  For flow network $G = (V,E)$, in which each edge $(u,v) \in E$ has non-negative
  integer capacity $c(u,v) \geq 0$, maximum flow is an integer.
\end{theorem}

\begin{proof}
  By Max-flow min-cut theorem \cite[Theorem 26.6][page 723]{introtoalg}\\
  If $f$ is a flow in a flow network $G = (V,E)$ with source $s$ and sink $t$, then the
  following conditions are equivalent:

  \begin{enumerate}
    \item $f$ is a maximum flow in $G$.
    \item The residual network $G_f$ contains no augmenting paths.
    \item $|f| = c(S,T)$ for some cut $(S,T)$ of $G$, where
  \end{enumerate}
  $$c(S,T) = \sum_{u \in S} \sum_{v \in T} c(u,v)$$

  As soon as integer set is closed under the operation of addition, thus
  $c(S,T)$ is integer and maximum flow in a flow network $G$ is integer as well. 
\end{proof}

\begin{theorem}
  For flow network $G = (V,E)$, in which each edge $(u,v) \in E$ has non-negative
  integer capacity $c(u,v) \geq 0$, if $C=\sum\{c(s,q)|q \in V\}$,
  then the Ford-Fulkerson algorithm runs in time $O(C \cdot m)$, where m is the number of edges.
\end{theorem}

\begin{proof}

\cite[Analysis of Ford-Fulkerson algorithm][page 725]{introtoalg}\\

Consider Ford-Fulkerson algorithm:\\

\IncMargin{1em}
\begin{algorithm}[H]
  \SetAlgoNoEnd\SetAlgoNoLine
  \SetNlSkip{1.5em}
  \DontPrintSemicolon

  \nonl \textsc{Ford-Fulkerson}$(G,s,t)$\;
  \ForEach{$edge (u,v) \in G.E$}{
    $(u,v).f = 0$\;
  }

  \While {there exists a path $\rho$ from $s$ to $t$ in residual network $G_f$}{
    $c_f(\rho) = \text{min}\{c_f(u,v):(u,v) is in \rho\}$
    \ForEach{$edge (u,v) \in \rho$}{
      \eIf{$(u,v) \in E$}{
        $(u,v).f = (u,v).f + c_f(\rho)$
      }{
        $(v,u).f = (v,u).f - c_f(\rho)$
      }
    }
  }
  \Return edge belongs to some MST\;
\end{algorithm}

\vspace{1em}

As soon as $C$ represent bound of maximum flow in cut $(S,T)$, where
$S=\{s\}$, thus maximum flow less or equal $C$. Therefore a straightforward
implementation executes the \textbf{while} loop of lines 3-8 at most $C$ times,
since the flow value increases by at least one unit in each iteration.
We can perform the work done within the \textbf{while} loop efficiently if we implement
the flow network $G=(V,E)$ with the right data structure and find an augmenting
path by a linear-time algorithm.
Let us assume that we keep a data structure corresponding to a directed graph
$G^\prime=(V,E^\prime)$, where $E^\prime=\{(u,v):(u,v) \in E \text{ or } (v,u)
\in E\}$. Edges in the network $G$ are also edges in $G^\prime$, and therefore we can
easily maintain capacities and flows in this data structure.
Given a flow $f$ on $G$,
the edges in the residual network $G$ $f$ consist of all edges $G_f$ such that
$c_f(u,v) > 0$, where $c_f$ conforms to equation:

$c_f = 
\begin{cases}
  c(u,v)-f(u,v) & \text{if } (u,v) \in E \\
  f(v,u)        & \text{if } (v,u) \in E \\
  0             & \text{otherwise} \\
\end{cases}
$

The time to find a path in a residual network is therefore $O(|V| + |E^\prime|)$
if we use either depth-first search or breadth-first search. Each iteration of
the \textbf{while} loop thus takes $O(|E|)$ time, as does the initialization in lines
1–2, making the total running time of the \textsc{Ford-Fulkerson} algorithm
$O(|E| \cdot C)$ 
\end{proof}

\textbf{Exercise 3.2} Phones and base-stations

Naive algorithm for solving this problem - for each base-station $i \in K$ calculate
distances to phone $j \in N$, then push to max-heap of station $i$ by
distance (except those what are further than max-distance $D$).

The worst case - $O(K \cdot N \cdot \text{log} N)$.

Then, for each station $i$ take one phone $j$ from top of heap and mark phones as
connected. If phone was already connected just take next one (in the worst case
this gives another multiplication by K in complexity). Repeat until load-factor of
stations reaches the limit (L iterations). Done.

The worst case - $O(K^2 \cdot L \cdot \text{log} N)$.

But this algorithm doesn't connect all possible clients.\\

Another naive algorithm for solving this problem - for each phone/station construct
reachable matrix with total count of reachable station.\\

\begin{tabular}{|c|c|c|c|c|c|}
  \hline
        & $s_1$ & $s_2$ & $\dots$ & $s_n$ & total\\
  \hline
  $p_1$ & 0 & 1 & \dots & 1 & 2\\
  $p_2$ & 0 & 1 & \dots & 0 & 1\\
  \dots & \dots & \dots & \dots & \dots & \dots\\
  $p_n$ & 1 & 0 & \dots & 0 & k\\
  \hline
\end{tabular}\\

Time complexity $O(N \cdot K)$, space complexity $O(N \cdot K)$.\\

Then construct order min-heap by count of reachable stations.

Time complexity $O(N \text{ log}N)$, space complexity $O(N)$.\\

Then extract phone and connect to not fully load station.

Time complexity $O(N \text{ log}N \cdot K)$ (another multiplication from search
and check of available station in array)\\

This one should provide a better result.\\

\bibliography{hw1}

\end{document}